
\documentclass{article}
\usepackage{graphicx}
\usepackage{listings}
\usepackage{geometry}
\geometry{margin=1in}
\begin{document}

\title{Practical Work 1: TCP File Transfer}
\author{Report}
\date{03/12/2025}
\maketitle

\section{Introduction}
The goal of Practical Work 1 is to implement a 1--1 file transfer system over TCP/IP using low-level socket programming.
The system contains two components: a server that receives files and a client that sends files.

\section{Protocol Design}
Our custom protocol consists of exchanging metadata before sending the actual file content.

\subsection*{Protocol Steps}
\begin{itemize}
    \item Client sends filename length, filename, and file size.
    \item Client streams the file content in fixed-size chunks.
    \item Server writes received content to disk and replies ``OK''.
\end{itemize}

\subsection*{Protocol Diagram}
\begin{verbatim}
 Client                                  Server
   |                                        |
   |------ connect() ---------------------->|
   |-- send(filename_length) -------------->|
   |-- send(filename) --------------------->|
   |-- send(file_size) -------------------->|
   |-- send(file_content) ----------------->|
   |<------------ "OK" ---------------------|
   |------------- close() ----------------->|
\end{verbatim}

\section{System Architecture}
The system follows a client--server model.

\subsection*{Architecture Diagram}
\begin{verbatim}
+-----------------+         TCP         +-----------------+
|     Client      |  -----------------> |     Server      |
|  - open file    |                    | - accept()       |
|  - send file    | <-----------------  | - write to disk |
+-----------------+    ACK/OK msg      +-----------------+
\end{verbatim}

\section{Implementation}

\section{Server Code}
\begin{lstlisting}[language=C]
#include <stdio.h>
#include <stdlib.h>
#include <string.h>
#include <winsock2.h>
#include <ws2tcpip.h>

#define PORT 5000
#define BUFFER_SIZE 4096

int main() {
    WSADATA wsa;
    SOCKET server_fd, client_fd;
    struct sockaddr_in address;
    int addrlen = sizeof(address);
    char buffer[BUFFER_SIZE];

    WSAStartup(MAKEWORD(2,2), &wsa);

    server_fd = socket(AF_INET, SOCK_STREAM, 0);

    address.sin_family = AF_INET;
    address.sin_addr.s_addr = INADDR_ANY;
    address.sin_port = htons(PORT);

    bind(server_fd, (struct sockaddr*)&address, sizeof(address));
    listen(server_fd, 1);

    client_fd = accept(server_fd, NULL, NULL);

    int name_len = 0;
    recv(client_fd, (char*)&name_len, sizeof(name_len), 0);

    char filename[256] = {0};
    recv(client_fd, filename, name_len, 0);

    long filesize = 0;
    recv(client_fd, (char*)&filesize, sizeof(filesize), 0);

    FILE *fp = fopen(filename, "wb");

    long received = 0;
    while (received < filesize) {
        int n = recv(client_fd, buffer, sizeof(buffer), 0);
        fwrite(buffer, 1, n, fp);
        received += n;
    }

    send(client_fd, "OK", 2, 0);
    fclose(fp);
}
\end{lstlisting}

\section{Client Code}
\begin{lstlisting}[language=C]
#include <stdio.h>
#include <stdlib.h>
#include <string.h>
#include <winsock2.h>
#include <ws2tcpip.h>

#define PORT 5000
#define BUFFER_SIZE 4096

int main() {
    WSADATA wsa;
    SOCKET sock;
    struct sockaddr_in serv_addr;
    char filename[256];
    char buffer[BUFFER_SIZE];

    printf("Enter filename to send: ");
    scanf("%s", filename);

    FILE *fp = fopen(filename, "rb");

    WSAStartup(MAKEWORD(2,2), &wsa);

    sock = socket(AF_INET, SOCK_STREAM, 0);

    serv_addr.sin_family = AF_INET;
    serv_addr.sin_port = htons(PORT);
    inet_pton(AF_INET, "127.0.0.1", &serv_addr.sin_addr);

    connect(sock, (struct sockaddr*)&serv_addr, sizeof(serv_addr));

    int name_len = strlen(filename);
    send(sock, (char*)&name_len, sizeof(name_len), 0);
    send(sock, filename, name_len, 0);

    fseek(fp, 0, SEEK_END);
    long filesize = ftell(fp);
    send(sock, (char*)&filesize, sizeof(filesize), 0);
    fseek(fp, 0, SEEK_SET);

    int bytes;
    while ((bytes = fread(buffer, 1, BUFFER_SIZE, fp)) > 0) {
        send(sock, buffer, bytes, 0);
    }

    char ack[10] = {0};
    recv(sock, ack, sizeof(ack), 0);
}
\end{lstlisting}

\end{document}
\section{Conclusion}
The TCP file transfer system demonstrates a basic but effective implementation of network communication using sockets, metadata exchange, and binary file transfer.

\end{document}
